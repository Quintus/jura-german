\documentclass[11pt,a4paper,DIV=calc]{scrartcl}
\usepackage[ngerman]{babel}
\usepackage{csquotes}
\usepackage[backend=biber,style=rub-jura]{biblatex}
\usepackage{fontspec}
\usepackage{xspace}
\usepackage{xcolor}
\usepackage[hidelinks]{hyperref}

%\addbibresource{realrefs.bib}
\addbibresource[label=example]{examples.bib}

\setmainfont{Linux Libertine O}
\setsansfont{Linux Biolinum O}

%\definecolor{darkblue}{rgb}{03039d}

\renewcommand\thesection{\Roman{section}}
\renewcommand\thesubsection{\arabic{subsection}}

\addtokomafont{section}{\color{blue}}

\newcommand\software[1]{\textsf{#1}}
\newcommand\rj{\software{rub-jura}\xspace}
\newcommand\Biblatex{\software{Bib\LaTeX{}}\xspace}
\newcommand\name[1]{\textit{#1}}
\newcommand\zB{z.\,B.\xspace}
\AtBeginDocument{\renewcommand\dh{d.\,h.\xspace}}

\newenvironment{rubexample}{\par\vspace{\baselineskip}\hrule\par\begin{refsection}}{\end{refsection}\par\hrule\par\vspace{\baselineskip}}

\KOMAoptions{DIV=last}

\begin{document}
\title{rub-jura}
\subtitle{Juristischer Bibliografierstil}
\author{Marvin Gülker}
\date{\today}
\maketitle{}

\tableofcontents{}

\section{Einführung}

Bei \rj handelt es sich um einen Bibliografier- und Zitierstil für das
\Biblatex-System. Er ist nicht für das veraltete \software{Bib\TeX{}}
gedacht und funktioniert mit diesem auch nicht.

Ich habe diesen Stil geschrieben, weil keine vorhandenen Stile den mir
gemachten Vorgaben entsprachen. Dicht heran kam das Paket
\software{biblatex-juradiss} von Dr. \name{Tobias Pfeiffer}, welches
aber nicht mehr gepflegt wird und zudem zusätzlich von
\software{biblatex-dw} abhängt. Der vorliegende Stil ist unabhängig
von dritten Paketen und hängt nur von \Biblatex selbst ab.

Wie in ähnlich gelagerten Stilen tauchen bei der Zitierweise deutscher
Juristen gewisse Probleme auf, die man am besten so beschreiben kann:
Sie ist überhaupt nicht einheitlich. Selbst innerhalb eines
Eintragstyps werden manche Werke gänzlich anders zitiert als andere,
als Beispiel seien Gerichtsentscheidungen genannt, die normalerweise
mit dem Gericht zuerst zitiert werden — außer die Entscheidung wurde
in einer amtlichen Sammlung veröffentlicht. Dann tritt diese an die
Stelle des Gerichts. Außer im Literaturverzeichnis, wo dies nicht ins
Gewicht fällt. Kommentare als spezifisch juristische Literatur sind
ohnehin ein Spezialfall.

\section{Einschränkungen}

Dieser Stil funktioniert für mich, was aber noch lange nicht heißt,
dass er für andere Leute geeignet ist. Er kommt den Vorgaben an der
juristischen Fakultät der Ruhruniversität Bochum nach. Abgesehen von
diesen äußeren Beschränkungen gibt es aber einige handfeste technische
Probleme, die sich nicht ohne weiteres lösen lassen:

\begin{itemize}
\item Bei Angabe der Option \verb+morejurisdict+ ist das
  \verb+postnote+-Feld eines \verb+@jurisdiction+-Eintrags nicht
  sinnvoll als bloße Seitenangabe verwendbar, da nicht klar ist,
  worauf die Zahl sich bezieht.
\item Für Zitate aus Einträgen des Typs \verb+@commentary+ hat das
  \verb+prenote+-Feld eine besondere Bedeutung: Es gibt den Bearbeiter
  an. Es steht — auch bei Kommentaren, die nur einen Bearbeiter haben,
  wie etwa dem \emph{Fischer} zum Strafgesetzbuch — nicht für andere
  Anmerkungen zur Verfügung und wird bei Nutzung automatisch mit einem
  nachfolgenden „in:“ versehen.
\item Die \verb+ibidem+-Funktion ist nur etwas für Mutige. Juristen
  kennen keine „ebd.“-Angabe, sondern arbeiten mit einer eigenwilligen
  Fassung von „a.\,a.\,O.“, die in diesem Stil sogar grundlegend umgesetzt
  wurde („\textit{Muscheler} a.\,a.\,O.“). Sie schlägt jedoch für Einträge
  des Typs \verb+@commentary+ fehl.

  Ich rate von der Nutzung daher für den Moment ab. Wer es dennoch
  wagen will, sollte sowohl den \verb+ibid+- als auch den
  \verb+loccit+-Tracker aktivieren.
\item \verb+\multinamedelim+ und \verb+\finalnamedelim+ können nicht
  gefahrlos überschrieben werden, da einige Typen an bestimmten
  Stellen das Makro intern neu setzen. Der Nutzer dieses Stils wird
  sich also mit dem Standardaufbau von Namenslisten zufrieden geben
  müssen (oder so tief in die Interna einsteigen, dass er die
  betroffenen Bibmakros ändern kann).
\end{itemize}

Der Stil ist zurzeit „Work in Progress“ und wohl noch lange nicht
fertig. Es ist daher durchaus möglich, dass die obigen Restriktionen
noch aufgehoben werden. Auch weise ich darauf hin, dass die minutiöse
Anpassung der Fußnoten auf die jeweiligen Eintragstypen nicht
vollständig ist, was ich aber gern noch nacharbeiten will. Schließlich
ist die große Menge von \Biblatex unterstützter Eintragstypen einer
seiner besonderen Vorteile. Unterstützt werden zurzeit die in Tabelle
\ref{tab:unterstuetzte-typen} aufgeführten primären Eintragstypen;
andere Typen können funktionieren, müssen aber nicht. Daneben gibt es
noch einige Aliastypen, \dh Typen, die selbst nur Verweise auf einen
anderen Typ sind, die aber aus logischen Gründen vorzuziehen
sind. Diese werden in Tabelle \ref{tab:aliastypen} aufgelistet; für
weitere Informationen diesbezüglich sei auf die
\Biblatex-Do\-ku\-men\-ta\-tion verweisen.

\begin{table}
  \centering

  @article @book @commentary @inbook @incollection @inproceedings @jurisdiction @manual @misc

  \caption{Unterstützte Eintragstypen}
  \label{tab:unterstuetzte-typen}
\end{table}

\begin{table}
  \centering
  \begin{tabular}{ll}
    \textbf{Name} & \textbf{Alias für}\\
    \hline
    @mvbook         & @book\\
    @bookinbook     & @book\\
    @suppbook       & @book
  \end{tabular}
  \caption{Aliastypen}
  \label{tab:aliastypen}
\end{table}

\section{Inkompatibilitäten}\label{sec:inkompat}

Die ungewöhnliche Zitierweise deutscher Juristen bringt es mit sich,
dass dieser Stil nur begrenzt kompatibel mit anderen Stilen ist. Ich
habe mich sogar gezwungen gesehen, von \Biblatex{}s erweiterten
Funktionalitäten Gebrauch zu machen und neue Felder und Optionen für
die Bibliografiedateien zu definieren. Dies betrifft vornehmlich den
Eintragstyp \verb+@jurisdiction+, der im Standardreportoire von
\Biblatex aber ohnehin nur als „nicht unterstützt“ aufgeführt wird und
dementsprechend wenig Gefahr läuft, mit einem anderen Stil zu
kollidieren. Ähnlich liegt es bei der Auswertung der Felder eines
\verb+@commentary+.

Dieser Stil greift auf Funktionalitäten von \Biblatex zurück, die
ausschließlich mit \software{Biber} als Backend gewährleistet
sind. Ein Einsatz von \software{Bib\TeX{}} verbietet sich somit.

Außerdem definiert dieser Stil einige neue Paketoptionen, die von
anderen Stilen nicht verstanden werden:

\begin{itemize}
\item \verb+morejurisdict+
\item \verb+cmtryedit+
\end{itemize}

\section{Einbindung}

Es handelt sich um einen \Biblatex-Stil, daher erfolgt die Einbindung
wie bei jedem anderen \Biblatex-Stil auch.

\begin{verbatim}
\usepackage[backend=biber,style=rub-jura]{biblatex}
\end{verbatim}

\section{Paketoptionen}

Dieses Paket definiert einige neue Paketoptionen, die beim Latex des
\verb+biblatex+-Pakets angegeben werden können.

\begin{description}
\item[morejurisdict] Wird ein \verb+@jurisdiction+-Eintrag zitiert, so
  wird die Fußnote mit erweiterten Informationen (Sekundäre Fundstelle
  und Titel, so weit verfügbar) angereichert. In der Folge kann die
  Verwendung des \verb+postnote+Feldes schwierig lesbare Resultate
  hervorbringen.
\item[cmtryedit] Kommentare, die über einen \verb+shorthand+ zitiert
  werden, werden um die Auflage in Form einer hochgestellten
  Zahl\textsuperscript{42} erweitert. Dies ist normalerweise
  überflüssig, kann aber bei der Zitierung aus unterschiedlichen
  Auflagen desselben Kommentars eine praktische Zusatzinformation
  sein.

  Kommentare ohne \verb+shorthand+ sind von dieser Option nicht
  betroffen. Das liegt daran, dass kein „zitiert als“-Eintrag im
  Literaturverzeichnis folgt, mit dem man die Zuordnung verdeutlichen
  könnte. Es handelt sich mithin um eine Designentscheidung, nicht um
  eine Fehlfunktion. Wer die Auflage bei allen Kommentaren setzen
  möchte, kann für jeden Kommentar einen \verb+shorthand+ etwa in Form
  der Autorennachnamen (etwa „Jarass/Pieroth“) definieren.
\end{description}

\section{Einzelne Eintragstypen}

Im Gegensatz zum Paket \verb+biblatex-juradiss+ versucht sich dieser
Stil möglichst dicht an den Bedeutungen der Eintragstypen wie sie in
der \Biblatex-Do\-ku\-men\-ta\-tion beschrieben sind zu halten. Damit sollte
der höchstmögliche Grad an Kompatibilität zu anderen Stilen (zu den
Ausnahmen siehe Abschnitt \ref{sec:inkompat}) gewährleistet sein. Für
die generelle Bedeutung der Typen sei dementsprechend auch auf die
\Biblatex-Do\-ku\-men\-ta\-tion verwiesen, hier aufgeführt werden nur die
besonderen Eigenarten juristischer Literatur.

\subsection{@artice}

Mit \verb+@article+ werden Publikationen in Fachzeitschriften und laut
\Biblatex-Do\-ku\-men\-ta\-tion auch Aritkel in Tageszeitungen erfasst (auch
wenn ich letzteres für unglücklich halte und einen separaten
Eintragstyp befürworte). Artikel werden stets nach Seitenzahl und
niemals nach Randnummer zitiert; Einträge dieses Typs geben in der
Fußnote daher immer die Startseite des Artikels aus, unabhängig davon,
ob man zusätzlich eine Fundstellenseite über das \verb+postnote+-Feld
angibt.

\begin{rubexample}
\begin{verbatim}
@article{hoernle-huster,
  author = "Tatjana Hörnle and Stefan Huster",
  title = "Wie weit reicht das Erziehungsrecht der Eltern?",
  subtitle = "Am Beispiel der Beschneidung von Jungen",
  journaltitle = "JZ",
  year = 2013,
  pages = "328-339"
}
\end{verbatim}

Auch andere Vertreter in der Literatur haben sich für diese
Herangehensweise ausgesprochen\footcite[330]{hoernle-huster}.

\printbibliography
\end{rubexample}

\subsection{@book}

Der Eintragstyp \verb+@book+ umfasst Monografien und Lehrbücher.

Es ist in der juristischen Fachliteratur üblich, insbesondere die
meist nur das Rechtsgebiet benennenden Titel von Lehrbüchern in den
Fußnoten durch einen Kurztitel zu ersetzen. Dies wird unterstützt, im
Literaturverzeichnis wird selbstverständlich der vollständige Titel
aufgeführt.

\begin{rubexample}

\begin{verbatim}
@book{muscheler,
  title = "Familienrecht",
  author = "Karlheinz Muscheler",
  edition = 3,
  publisher = "Franz Vahlen",
  year = 2013,
  location = "München"
}
@book{ipsen,
  title = "Staatsrecht I",
  subtitle = "Staatsorganisationsrecht",
  shorttitle = "StaatsR I",
  author = "Jörn Ipsen",
  edition = 25,
  publisher = "Franz Vahlen",
  year = 2013,
  location = "München"
}
\end{verbatim}

Der Inhalt des Begriffs „Familie“ ist aus dem Gesetz zu
beurteilen\footcite[\S\,1 Rnr. 3]{muscheler}.

Die Vertreter der Länder im Bundesrat können ihre Stimmen nur
einheitlich abgeben\footcite[\S\,7 Rnr. 343ff]{ipsen}.

\printbibliography
\end{rubexample}

\subsection{@commentary}

Der Typ \verb+@commentary+ wird von \Biblatex standardmäßig nur wie
ein Eintrag des Typs \verb+@misc+ behandelt. Dieser Stil dagegen
setzt einen eigenen \emph{Driver} ein, um der Bedeutung, die diese Art
von Literatur in der Jurisprudenz hat, auch gerecht zu werden. Leider
sind sie ebenso wichtig wie komplex, sodass es ein paar Dinge zu
beachten gibt.

Zunächst einmal werden manche Kommentare nach ihrem Begründer, andere
nach ihren Herausgebern, und wieder andere nach Eigennamen zitiert
(„Palandt“, „Jarass/Pieroth“, „Leipziger Kommentar zum
Strafgesetzbuch“). Es scheint nach den Beobachtungen des Autors im
wesentlichen drei Kategorien von Kommentaren zu geben:

\begin{enumerate}
\item Solche, die eher unbekannt sind. Sie werden in der Fußnote stets
  mit Autor bzw. Herausgeber und Titel bzw. Kurztitel zitiert.
\item Solche, die aufgrund ihrer Bekanntheit (und ihres Umfangs mit
  entsprechender Zahl der Bearbeiter) mit einem Eigennamen zitiert
  werden, der vollständig an die Stelle der Herausgeber tritt.
\item Solche, die nach ihrem Begründer zitiert werden, aber schon so
  bekannt sind, dass man den Titel bzw. Kurztitel nicht mehr angibt.
\end{enumerate}

Kommentare der dritten Kategorie sind es, die den Quellcode für
\verb+@commentary+ recht komplex werden lassen. Leider fällt mit
dem \name{Palandt} ein recht wichtiger Kommentar in diese Kategorie,
sie kann deshalb nicht einfach als Sonderfall abgeschrieben werden,
den man eben einfach gleich wie solche der ersten Kategorie behandeln
kann. Während allgemein erwartet wird dass man den \name{Fischer} als
„\name{Fischer}, StGB, §…“ zitiert (weil er in die erste Kategorie
fällt), stieße es doch auf Widerstand, den \name{Palandt} als
„\name{Palandt}, BGB, §…“ zu zitieren. Letzterer muss vielmehr wie ein
Eigennamenskommentar gleich dem \name{Münchener Kommentar} in der
Fußnote erscheinen: „\name{Palandt}, §…“, „\name{MüKo}, §…“ (natürlich
jeweils mit Bearbeiter vorweg). Im Literaturverzeichnis aber müssen
die Kommentare der zweiten Kategorie unter ihrem Eigennamen als
Langtitel („Münchener Kommentar zum Bürgerlichen Gesetzbuch“)
aufgeführt und einsortiert werden, Kommentare der dritten Kategorie,
obwohl doch zitiert als hätten sie Eigennamen, werden jedoch
einsortiert wie Kommentare der ersten Kategorie, \dh unter dem Namen
des Begründers einschließlich Vorname. Um dieser Komplexität Rechnung
zu tragen, führt dieser Stil eine eigene Eintragsoption für Einträge
des Typs \verb+@commentary+ ein:

\begin{description}
\item[citebytitle] Wird diese Option angegeben, wird der Titel des
  Kommentars als Eigenname angesehen und er wird unter diesem Titel in
  das Literaturverzeichnis eingepflegt. Fehlt sie, wird der Kommentar
  stattdessen unter dem Namen des Herausgebers eingepflegt.

  Diese Option sollte nur zusammen mit den Optionen
  \verb+useauthor=false+ und \verb+useeditor=false+ angegeben
  werden. Dies stellt sicher, dass \Biblatex den Eintrag nicht nur mit
  dem Titel voran formatiert, sondern ihn auch nach dem Titel
  einsortiert (und nicht etwa der \name{Erman} bei „W“ wie
  \name{Westermann} einsortiert wird).
\end{description}

Weiterhin zitiert man Eigennamenskommentare auch in den Fußnoten nicht
mit ihren Herausgebern, sondern mit dem Eigennamen in gekürzter
Form. Aus Sicht von \Biblatex handelt es sich dabei um eine vollstände
Ersetzung aller bibliografischen Angaben durch einen einzigen,
bekannten und kurzen Terminus. Für solche sieht \Biblatex das Feld
\verb+shorthand+ vor, von welchem man auch tunlichst Gebrauch machen
sollte, wenn man einen Kommentar der zweiten oder dritten Kategorie
bemüht. Die Verwendung des Felds \verb+shorthand+ ergänzt in der
Literaturliste den Eintrag um ein „zitiert als: \name{Bearbeiter}, in:
shorthand“ und ersetzt in der Fußnote die Herausgeberliste und den
Titel durch den \verb+shorthand+. Beispiele finden sich unten.

Schließlich erhält das Feld \verb+prenote+ bei der Zitation von
Kommentaren eine besondere Bedeutung, die auch nicht übergangen werden
kann. Dieses Feld gibt \emph{immer} den Bearbeiter an, und wird in der
Fußnote automatisch um ein „in:“ ergänzt. Für Kommentare, die nur
einen einzigen Bearbeiter haben, ist das zwar streng genommen nicht
erforderlich, doch ist es technisch zu aufwendig, diese Bedingung
abzuprüfen und davon abhängig das \verb+prenote+-Feld mit der
Sonderbedeutung zu belegen oder nicht. Außerdem wird so eine gewisse
Konsistenz erreicht, die zu einer einfach zu merkenden Faustregel
führt: \emph{Das \texttt{prenote}-Feld darf bei Kommentaren aller Art
  ausschließlich für die Angabe des Bearbeiters benutzt werden.} Bei
Kommentaren, die nur einen Bearbeiter haben, darf es folglich
\emph{gar nicht} benutzt werden. Das \verb+postnote+-Feld ist davon
unberührt und kann wie sonst auch für Anhängsel aller Art verwandt
werden.

Wird beim Laden des \software{biblatex}-Pakets die von diesem Stil
definierte Option \verb+cmtryedit+ angegeben, so wird an den
\verb+shorthand+ automatisch die Auflage als hochgestellte Zahl
angehangen.

Ähnlich wie bei \verb+@incollection+ (Abschnitt
\ref{sec:incollection}) ist die Formatierung eines Kommentars im
Literaturverzeichnis sehr komplex und nutzt unterschiedliche
Ausgabeschemata für Namen. Eine Anpassung ohne tiefergehende
Kenntnisse des Quellcodes dieses Stils scheidet aus.

\begin{rubexample}
\begin{verbatim}
@commentary{erman14,
  title = "Erman Bürgerliches Gesetzbuch, Handkommentar",
  shorthand = "Erman",
  editor = {Harm Peter Westermann and Barbara Grunewald and
    Georg Maier-Reimer},
  gender = "pm",
  edition = 14,
  location = "Köln",
  year = 2014,
  options = {citebytitle,useauthor=false,useeditor=false}
}
@commentary{palandt15,
  title = "Bürgerliches Gesetzbuch mit Nebengesetzen",
  editor = "Otto Palandt",
  editortype = "founder",
  shorthand = "Palandt",
  gender = "sm",
  edition = 74,
  year = 2015,
  location = "München",
}
@commentary{fischer62,
  title = "Strafgesetzbuch mit Nebengesetzen, Kommentar",
  shorttitle = "StGB",
  editor = "Thomas Fischer",
  gender = "sm",
  edition = 62,
  year = 2015,
  location = "München",
  publisher = "C.\,H. Beck"
}
@commentary{leipzig06,
  title = "Leipziger Kommentar zum Strafgesetzbuch",
  shorthand = "LK-StGB",
  editor = "Heinrich Wilhelm Laufhütte and
    Ruth Rissing-van Saan and Klaus Tiedemann",
  edition = 12,
  location = "Berlin",
  year = 2006,
  options = {citebytitle,useauthor=false,useeditor=false}
}
\end{verbatim}

All\footcite[Ellenberger][\S\,12 Rnr. 7]{palandt15}
überall\footcite[Grunewald][\S\,12 Rnr. 6]{erman14}
auf den
Tannenspitzen\footcites[][\S\,242 Rnr. 10]{fischer62}[Laufhütte][\S\,242\,Rnr. 15]{leipzig06}\ldots

\printbibliography
\end{rubexample}

\subsection{@inbook}

Gelegentlich kommt es vor, dass innerhalb eines Buches klar nach Autor
und Titel abgrenzbare Abschnitte vorliegen, ohne dass das Buch ein
Sammelband (etwa eine Festschrift) im Sinne einer \verb+@incollection+
wäre. Das ist typischerweise bei Anwaltshandbüchern der Fall. Für diese
Art von Literatur empfiehlt sich \verb+@inbook+, welches im Gegensatz
zu \verb+@incollection+ auch nicht zwangsweise mit einer Seitenangabe
zitiert werden muss (da \verb+@incollection+ stets die Startseite des
Beitrags in der Fußnote mit ausgibt, dieser Typ dies jedoch nicht
tut).

Für diesen Typ empfiehlt sich die Angabe des Feldes
\verb+chapter+. Sie führt (übrigens auch bei anderen Typen) dazu, dass
der dort angegebene Text vor der Seitenzahl im Literaturverzeichnis
landet und die Seitenzahl in Klammern dahinter. Ohne das Feld wird die
Seitenzahl ohne Klammern ausgegeben.

\begin{rubexample}
\begin{verbatim}
@inbook{handbuch,
  author = "Max Mustermann",
  title = "Verträge im Kauf- und Werkvertragsrecht",
  booktitle = "Anwaltshandbuch für diverses Recht",
  pages = "110-195",
  chapter = "\S\,23",
  editor = "Keiner Niemand and Sonst Wer",
  location = "Frankfurt an der Oder",
  year = 1999
}
\end{verbatim}

Die Praxis hält es ganz ähnlich\footcite[\S\,23 Rnr.\,55]{handbuch}.

\printbibliography
\end{rubexample}

\subsection{@incollection}\label{sec:incollection}

Bei einer \verb+@collection+ handelt es sich um einen Sammelband,
meist eine Fest- oder Gedächtnisschrift. Eher selten verweist man auf
eine solche als Ganze, sondern zitiert aus den einzelnen,
selbstständigen Beiträgen, die mit dem Typ \verb+@incollection+
erfasst werden. Dieser Typ weist eine besonders schwierige
Formatierung im Literaturverzeichnis auf, denn aus unbekannten Gründen
werden die Herausgeber einer Festschrift nicht wie sonst üblich mit
Schrägstrichen / getrennt, sondern vielmehr in einer Satzkonstruktion
mit Komma und „und“ aufgelistet. Intern nutzt dieser Typ daher eine
eigene Definition der Makros \verb+\multinamedelim+ und
\verb+\finalnamedelim+.

\emph{Keine} \verb+@incollection+ sind etwa Anwaltshandbücher, die
auch meist nicht — wie von diesem Typ vorgesehen — mit Seite sondern
mit Randnummer zitiert werden. Für sie wäre wohl \verb+@inbook+
angebrachter.

Da dieser Typ zwangsweise die Startseite des zitierten Artikels
ausgibt, kann er nicht für den Verweis auf nicht nach Seitenzahlen
zitierten Publikationen herangezogen werden.

Einträge des Typs \verb+@incollection+ müssen einen \verb+shorttitle+
haben, um korrekt und vorgabenkonform zu funktionieren. Alle Fest- und
Gedächtnisschriften besitzen einen solchen: „FS Geehrter“ bzw. „GS
Geehrter“.

\begin{rubexample}
\begin{verbatim}
@incollection{germann,
  author = "Michael Germann",
  title = "Der menschliche Körper als Gegenstand der
    Religionsfreiheit",
  pages = "35-58",
  editor = "Bernd-Rüdiger Kern and Hans Lilie",
  booktitle = "Jurisprudenz zwischen Medizin und Kultur.
    Festschrift zum 70. Geburtstag von Gerfried Fischer.",
  shorttitle = "FS Fischer",
  location = "Frankfurt am Main",
  year = 2010
}
\end{verbatim}

Ein Beispielsatz mit Beispieltext\footcite[41]{germann}.

\printbibliography
\end{rubexample}

\subsection{@inproceedings}

Der Typ \verb+@inproceedings+ erfasst Kompilationen von
Konferenzvorträgen. Er ist etwa einschlägig, wenn anlässlich eines
bestimmten Termins in umfangreichen Vorträgen über ein bestimmtes
Thema referiert wird und die jeweiligen Sprecher ihre Ergebnisse in
einem speziell auf die Konferenz zugeschnittenen Artikel
zusammenfassen, und diese Artikel von einem Herausgeber gesammelt und
als Konferenzband herausgegeben werden. Maßgebliches Kennzeichen
dieser Art von Publikation ist ihre enge Bindung an das
Veranstaltungsereignis, die sich auch in der Angabe zahlreicher die
Veranstaltung betreffender Felder (\verb+eventtitle+,
\verb+eventdate+, \verb+venue+, \verb+organization+) manifestiert.

Keine Konferenzen sind die Debatten der gesetzgebenden Organe und
ihrer Ausschüsse, hierfür ist der Eintragstyp \verb+@legislation+
vorzuziehen.

Für die Fußnoten bewegt sich dieser Typ zwischen \verb+@incollection+
und \verb+@inbook+. Genauso wie der erstere kann dieser Typ nur mit
Seitenzahl zitiert werden (weil die Startseite automatisch mit
ausgegeben wird), weist aber nicht auch dessen Erfordernis eines
\verb+shorttitle+ auf. Vielmehr wird der gewöhnliche \verb+booktitle+
genutzt.

\begin{rubexample}
\begin{verbatim}
@inproceedings{konferenzband,
  eventtitle = "8. Bayreuther Forum für Wirtschafts-
    und Medienrecht",
  eventdate = "2012-01-27/2012-01-28",
  venue = "Bayreuth",
  organization = "Forschungsstelle für Wirtschafts-
    und Medienrecht an der Universität Bayreuth",
  booktitle = "Der Schutz des Geistigen Eigentums
    im Internet",
  editor = "Stefan Leible",
  date = "2012",
  location = "Tübingen",
  author = "Jan Eichelberger",
  title = "Urheberrecht und Streaming",
  pages = "17-46",
}
\end{verbatim}

Neuerlich kommt dieses Thema immer stärker zur
Sprache\footcite[25]{konferenzband}.

\printbibliography
\end{rubexample}

\subsection{@jurisdiction}

Ähnlich schwierig wie \verb+@commentary+ war die Implementierung von
\verb+@jurisdiction+. Das Besondere an Rechtsprechung ist, dass sie
üblicherweise nicht in das Literaturverzeichnis aufgenommen wird;
dieser Stil unterstützt das aber nichtsdestotrotz. Es ist bei der
Verwendung dieses Typs daher darauf zu achten, den Aufruf von
\verb+\printbibliography+ so zu gestalten, dass er Einträge des Typs
\verb+@jurisdiction+ ausschließt:

\begin{verbatim}
\printbibliography[nottype=jurisdiction]
\end{verbatim}

Dieser Typ benötigt weitaus mehr Informationen, als sich mit den von
\Biblatex unterstützten Standardfeldern erfassen lässt. Daher hat der
Autor für diesen Typ \emph{zusätzliche} Felder definiert, die von
\Biblatex \emph{sonst nicht unterstützt} werden. Diese zusätzlichen
Felder lauten:

\begin{description}
\item[az] \emph{(literal)} Das Aktenzeichen des Urteils.
\item[primary] \emph{(literal, erforderlich)} Primäre
  Fundstelle. Hierbei kann es sich entweder um eine Fundestelle in der
  amtlichen Sammlung des erkennenden Gerichtes („BGHZ 123, 45“) oder
  um eine Fundstelle in einer der größeren juristischen
  Fachzeitschriften handeln („NJW 1999, 1111“). Gibt es eine
  Fundstelle in der amtlichen Sammlung, ist diese \emph{immer} die
  primäre Fundstelle.
\item[secondary] \emph{(literal)} Sekundäre Fundstelle. Zusätzlich zur
  primären Fundstelle kann das Urteil außerdem in der hier angegebenen
  Zeitschrift gefunden werden (\zB „FamRZ 2000, 2222“). Die sekundäre
  Fundstelle wird von diesem Stil automatisch mit einem
  Gleichheitszeichen = eingeleitet.
\end{description}

Die sonst zulässigen Felder werden von \Biblatex unterstützt, sodass
ihre Typen der \Biblatex-Do\-ku\-men\-ta\-tion entnommen werden können, sofern
hier nichts spezielleres angegeben ist. Sie lauten:

\begin{description}
\item[author] \emph{(erforderlich)} Das erkennende Gericht („BGH“, „AG Klagenfurt“)
\item[date] \emph{(erforderlich)} Datum der Entscheidung.
\item[options] Siehe dazu unten.
\item[title] Manche Entscheidungen werden vom erkennenden Gericht oder
  der interessierten Fachöffentlichkeit mit einem mehr oder weniger
  geistreichen Titel versehen („Herrenreiter“, „Elfes“,
  „Bearshare“). Dieser kann hier angegeben werden und wird dann im
  Literaturverzeichnis automatisch mit einem
  vorangehenden Halbgeviertstrich angeführt.
\item[type] \emph{(erforderlich)} „Urteil“, „Beschluss“ oder
  „Entscheidung“.
\end{description}

Der primären Fundstelle ist nicht anzusehen, ob sie amtlich ist oder
nicht, was aber für die Zitation ein wesentliches Kriterium ist: denn
ist die Fundstelle amtlich, wird die Entscheidung statt mit dem
Gerichtsnamen mit dem Namen der amtlichen Sammlung zitiert („BGHZ\ldots“
statt „BGH BGHZ\ldots“). Um diese Information zu erfassen, stellt
dieser Stil eine Option \verb+primamtl+ bereit. Enthält ein Eintrag
des Typs \verb+@jurisdiction+ diese Option, so wird die primäre
Fundstelle als amtliche behandelt. Ein Beispiel findet sich unten.

Abhängig von der Paketoption \verb+morejurisdict+ sind die Fußnoten zu
den Entscheidungen mehr oder weniger ausführlich. Fehlt die Option,
beschränkt sich die Fußnote auf das erkennende Gericht bzw. die
amtliche Sammlung und die Startseite, wird sie angegeben, dann werden
die sekundäre Fundstelle und der Titel ebenfalls gesetzt. Letzteres
verhindert leider den effektiven Einsatz des \verb+postnote+-Feldes,
da eine dort angegebene Seitenzahl dann einsam hinter dem Titel einer
Entscheidung landet. Kommt man freilich ohne Angabe einer exakten
Fundstellenseite aus, so kann man die Option \verb+morejurisdict+
getrost aktivieren und erhält so erheblich elegantere Fußnoten.

\begin{rubexample}
\begin{verbatim}
@jurisdiction{lg-koeln-njw-2012-2128,
  author = {{LG Köln}},
  gender = "sn",
  date = "2012-05-07",
  type = "Urteil",
  az = "Ns 169/11",
  primary = "NJW 2012, 2128",
  secondary = "FamRZ 2012, 1421"
}
@jurisdiction{bverfge-24-236,
  author = "BVerfG",
  gender = "sn",
  date = "1968-10-16",
  type = "Entscheidung",
  az = "1 BvR 241/66",
  primary = "BVerfGE 24, 236",
  secondary = "NJW 1967, 1234",
  title = "Fiktivtitel",
  options = {primamtl}
}
\end{verbatim}

Das haben auch schon diverse\footcite[2129]{lg-koeln-njw-2012-2128}
Gerichte erkannt\footcite{bverfge-24-236}.

\printbibliography
\end{rubexample}

\subsection{@manual}

Als \verb+@manual+ sind technische Dokumentationen,
Bedienungs- und Gebrauchsanleitungen und ähnliche Dokumente zu
qualifizieren. Auch die Dokumentationen von \LaTeX{}-Pa\-ke\-ten
fallen hierunter.

\begin{rubexample}
\begin{verbatim}
@manual{biblatex,
  author = "Philipp Lehman and Philip Kime and Audrey Boruvka
    and Joseph Wright",
  title = "The Biblatex package",
  subtitle = "Programmable Bibliographies and Citations",
  gender = "pm",
  date = "2015-04-20",
  version = "3.0.0",
  langid = "english",
  pagination = "section"
}
\end{verbatim}

\Biblatex unterstützt eine Menge verschiedener Felder für die
jeweiligen Eintragstypen\footcite[2.2.2]{biblatex}.

\printbibliography
\end{rubexample}

\end{document}

%%% Local Variables:
%%% mode: latex
%%% TeX-master: t
%%% End:
