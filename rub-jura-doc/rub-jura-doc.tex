\documentclass[11pt,a4paper,DIV=calc]{scrartcl}
\usepackage[ngerman]{babel}
\usepackage[backend=biber,style=rub-jura]{biblatex}
\usepackage{fontspec}
\usepackage{listings}
\usepackage{xspace}
\usepackage{xcolor}
\usepackage[hidelinks]{hyperref}

\setmainfont{Linux Libertine O}
\setsansfont{Linux Biolinum O}

%\definecolor{darkblue}{rgb}{03039d}

\renewcommand\thesection{\Roman{section}}
\renewcommand\thesubsection{\arabic{subsection}}

\addtokomafont{section}{\color{blue}}
\lstloadlanguages{[latex]tex}
\lstset{basicstyle=\ttfamily,language=tex}

\newcommand\software[1]{\textsf{#1}}
\newcommand\rj{\software{rub-jura}\xspace}
\newcommand\Biblatex{\software{Bib\LaTeX{}}\xspace}
\newcommand\name[1]{\textit{#1}}

\KOMAoptions{DIV=last}

\begin{document}
\title{ruby-jura-doc}
\subtitle{Juristischer Bibliografierstil}
\author{Marvin Gülker}
\date{\today}
\maketitle{}

\tableofcontents{}

\section{Einführung}

Bei \rj handelt es sich um einen Bibliografier- und Zitierstil für das
\Biblatex-System. Er ist nicht für das veraltete \software{Bib\TeX{}}
gedacht und funktioniert mit diesem auch nicht.

Ich habe diesen Stil geschrieben, weil keine vorhandenen Stile den mir
gemachten Vorgaben entsprachen. Dicht heran kam das Paket
\software{biblatex-juradiss} von Dr. \name{Tobias Pfeiffer}, welches
aber nicht mehr gepflegt wird und zudem zusätzlich von
\software{biblatex-dw} abhängt. Der vorliegende Stil ist unabhängig
von dritten Paketen und hängt nur von \Biblatex selbst ab.

Wie in ähnlich gelagerten Stilen tauchen bei der Zitierweise deutscher
Juristen gewisse Probleme auf, die man am besten so beschreiben kann:
Sie ist überhaupt nicht einheitlich. Selbst innerhalb eines
Eintragstyps werden manche Werke gänzlich anders zitiert als andere,
als Beispiel seien Gerichtsentscheidungen genannt, die normalerweise
mit dem Gericht zuerst zitiert werden — außer die Entscheidung wurde
in einer amtlichen Sammlung veröffentlicht. Dann tritt diese an die
Stelle des Gerichts. Außer im Literaturverzeichnis, wo dies nicht ins
Gewicht fällt. Kommentare als spezifisch juristische Literatur sind
ohnehin ein Spezialfall.

\section{Einschränkungen}

Dieser Stil funktioniert für mich, was aber noch lange nicht heißt,
dass er für andere Leute geeignet ist. Er kommt den Vorgaben an der
juristischen Fakultät der Ruhruniversität Bochum nach. Abgesehen von
diesen äußeren Beschränkungen gibt es aber einige handfeste technische
Probleme, die sich nicht ohne weiteres lösen lassen:

\begin{itemize}
\item Bei Angabe der Option \verb+morejurisdict+ ist das
  \verb+postnote+-Feld eines \verb+@jurisdiction+-Eintrags nicht
  sinnvoll als bloße Seitenangabe verwendbar, da nicht klar ist,
  worauf die Zahl sich bezieht.
\item Für Zitate aus Einträgen des Typs \verb+@commentary+ hat das
  \verb+prenote+-Feld eine besondere Bedeutung: Es gibt den Bearbeiter
  an. Es steht — auch bei Kommentaren, die nur einen Bearbeiter haben,
  wie etwa dem \emph{Fischer} zum Strafgesetzbuch — nicht für andere
  Anmerkungen zur Verfügung und wird bei Nutzung automatisch mit einem
  nachfolgenden „in:“ versehen.
\item Die \verb+ibidem+-Funktion ist nur etwas für Mutige. Juristen
  kennen keine „ebd.“-Angabe, sondern arbeiten mit einer eigenwilligen
  Fassung von „a.\,a.\,O.“, die in diesem Stil sogar grundlegend umgesetzt
  wurde („\textit{Muscheler} a.\,a.\,O.“). Sie schlägt jedoch für Einträge
  des Typs \verb+@commentary+ fehl.

  Ich rate von der Nutzung daher für den Moment ab. Wer es dennoch
  wagen will, sollte sowohl den \verb+ibid+- als auch den
  \verb+loccit+-Tracker aktivieren.
\end{itemize}

Der Stil ist zurzeit „Work in Progress“ und wohl noch lange nicht
fertig. Es ist daher durchaus möglich, dass die obigen Restriktionen
noch aufgehoben werden. Auch weise ich darauf hin, dass die minutiöse
Anpassung der Fußnoten auf die jeweiligen Eintragstypen nicht
vollständig ist, was ich aber gern noch nacharbeiten will. Schließlich
ist die große Menge von \Biblatex unterstützter Eintragstypen einer
seiner besonderen Vorteile. Unterstützt werden zurzeit die in Tabelle
\ref{tab:unterstuetzte-typen} aufgeführten Eintragstypen.

\begin{table}
  \centering\ttfamily

  @article @commentary @book @incollection @jurisdiction @misc

  \caption{Unterstützte Eintragstypen}
  \label{tab:unterstuetzte-typen}
\end{table}

\section{Inkompatibilitäten}\label{sec:inkompat}

Die ungewöhnliche Zitierweise deutscher Juristen bringt es mit sich,
dass dieser Stil nur begrenzt kompatibel mit anderen Stilen ist. Ich
habe mich sogar gezwungen gesehen, von \Biblatex{}s erweiterten
Funktionalitäten Gebrauch zu machen und neue Felder und Optionen für
die Bibliografiedateien zu definieren. Dies betrifft vornehmlich den
Eintragstyp \verb+@jurisdiction+, der im Standardreportoire von
\Biblatex aber ohnehin nur als „nicht unterstützt“ aufgeführt wird und
dementsprechend wenig Gefahr läuft, mit einem anderen Stil zu
kollidieren. Ähnlich liegt es bei der Auswertung der Felder eines
\verb+@commentary+.

Dieser Stil greift auf Funktionalitäten von \Biblatex zurück, die
ausschließlich mit \software{Biber} als Backend gewährleistet
sind. Ein Einsatz von \software{Bib\TeX{}} verbietet sich somit.

Außerdem definiert dieser Stil einige neue Paketoptionen, die von
anderen Stilen nicht verstanden werden:

\begin{itemize}
\item \verb+morejurisdict+
\item \verb+cmtryedit+
\end{itemize}

\section{Einbindung}

Es handelt sich um einen \Biblatex-Stil, daher erfolgt die Einbindung
wie bei jedem anderen \Biblatex-Stil auch.

\begin{lstlisting}
\usepackage[backend=biber,style=rub-jura]{biblatex}
\end{lstlisting}

\section{Paketoptionen}

Dieses Paket definiert einige neue Paketoptionen, die beim Latex des
\verb+biblatex+-Pakets angegeben werden können.

\begin{description}
\item[morejurisdict] Wird ein \verb+@jurisdiction+-Eintrag zitiert, so
  wird die Fußnote mit erweiterten Informationen (Sekundäre Fundstelle
  und Titel, so weit verfügbar) angereichert. In der Folge kann die
  Verwendung des \verb+postnote+Feldes schwierig lesbare Resultate
  hervorbringen.
\item[cmtryedit] Kommentare, die über einen \verb+shorthand+ zitiert
  werden, werden um die Auflage in Form einer hochgestellten
  Zahl\textsuperscript{42} erweitert. Dies ist normalerweise
  überflüssig, kann aber bei der Zitierung aus unterschiedlichen
  Auflagen desselben Kommentars eine praktische Zusatzinformation
  sein.

  Kommentare ohne \verb+shorthand+ sind von dieser Option nicht
  betroffen. Das liegt daran, dass kein „zitiert als“-Eintrag im
  Literaturverzeichnis folgt, mit dem man die Zuordnung verdeutlichen
  könnte. Es handelt sich mithin um eine Designentscheidung, nicht um
  eine Fehlfunktion. Wer die Auflage bei allen Kommentaren setzen
  möchte, kann für jeden Kommentar einen \verb+shorthand+ etwa in Form
  der Autorennachnamen (etwa „Jarass/Pieroth“) definieren.
\end{description}

\section{Einzelne Eintragstypen}

Im Gegensatz zum Paket \verb+biblatex-juradiss+ versucht sich dieser
Stil möglichst dicht an den Bedeutungen der Eintragstypen wie sie in
der \Biblatex-Dokumentation beschrieben sind zu halten. Damit sollte
der höchstmögliche Grad an Kompatibilität zu anderen Stilen (zu den
Ausnahmen siehe Abschnitt \ref{sec:inkompat}) gewährleistet sein.

\subsection{@book}



\end{document}

%%% Local Variables:
%%% mode: latex
%%% TeX-master: t
%%% End:
